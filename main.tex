
\newcommand*{\pointsize}{12pt}          %<Set the font size; make sure the size is correct
                                        %   for the font you will use
\documentclass[letterpaper,             % Use US letter-size paper
               oneside,                 % No verso and recto differences
               \pointsize]              % Uses the font size defined above
               {memoir}
\renewcommand{\cleardoublepage}%        % \cleardoublepage will create entirely blank
  {\clearpage}%                         %   pages depending on settings (e.g., usually
                                        %   before start of \mainmatter); redefine it here
                                        %   so that no entirely blank pages are created
                                        %   automatically
%%%%%%%%%%%%%%%%%%%%%%%%%%%%%%%%%%%%%%%
% (Some) Packages
%%%%%%%%%%%%%%%%%%%%%%%%%%%%%%%%%%%%%%%
\usepackage{graphicx}                   % For importing image files
\usepackage{etoolbox}                   % For advanced commands throughout preamble
\usepackage{microtype}                  %~Improves kerning and protrusion (optional);
                                        %   See here for an :
                                        %   http://www.khirevich.com/latex/microtype/

\providetoggle{usemicrotype}            % TRUE = microtype is being used
\makeatletter                           %   (Used to turn of microtype protrusion in the
\@ifpackageloaded{microtype}%           %   table of contents.)
  {\settoggle{usemicrotype}{true}}%
  {\settoggle{usemicrotype}{false}}
\makeatother
\usepackage{changepage}                 % For changing page layout (e.g., margins) in the
                                        %   middle of the document
\usepackage{calc}					              % Calculate text widths; used in page layout
                                        %   changes
\usepackage[hashEnumerators,smartEllipses]{markdown}
%%%%%%%%%%%%%%%%%%%%%%%%%%%%%%%%%%%%%%%
% Title page, input
%%%%%%%%%%%%%%%%%%%%%%%%%%%%%%%%%%%%%%%
\listadd{\titlelines}%
  {Data Driven Optimization of Reduced Neural Models}             %<Enter the title of the dissertation
\listadd{\titlelines}%
  {and Validation in the context of Large Scale Comparisons of Models and Data}            % If you want to
                                        %   split the title across lines,
                                        %   use another \listadd command for the
                                        %   second line
\newcommand*\Author{Russell Jarvis}          %<Enter your name; must match official transcript
\newcommand*{\documentname}%
  {Dissertation}                        %<Enter the type of document (capitalized)
\newcommand*{\degreename}
  {Doctor of Philosophy}                %<Enter the type of degree (capitalized)
\newcommand*\defdate{Month Year}        %<Give month (written out fully) and year of
                                        %   the oral defense
\listadd{\committeechair}{Chair Name}   %<Enter committee chair name; use \listadd for
%\listadd{\committeechair}{Another name}%   for additional names
\newcommand*{\chairlabel}{Prof S Crook}        %<If you have co-chairs, replace this text with
                                        %   'Co-Chair'
\listadd{\committeemember}{Prof R Gerkin} %<Enter committee member names; use \listadd for
\listadd{\committeemember}{Member Name} %   additional names
\newcommand*{\gradmonth}{Month}         %<Enter the graduation date; month can only be:
                                        %  May, August, or December
\newcommand*{\gradyear}{2020}           %<Enter the graduation year, e.g. 2014

\listadd{\keywords}{Optimization}          %<Enter keywords; use \listadd for
\listadd{\keywords}{Python, Neural Models}          %   additional names (up to 6)
\newcommand*{\graddate}{\gradmonth%     % Compose full graduation date
  \space\gradyear}

%%%%%%%%%%%%%%%%%%%%%%%%%%%%%%%%%%%%%%%
% Page layout
%%%%%%%%%%%%%%%%%%%%%%%%%%%%%%%%%%%%%%%
\settrimmedsize{\stockheight}%          % Specifies \paperheight and \paperwidth
  {\stockwidth}{*}
\settrims{0pt}{0pt}                     % Set location of page in relation to the stock.
                                        % Paper and stock size are equivalent,
                                        % so both \trimtop and \trimedge are set to 0pt
\newlength{\forfootskip}
\setlength{\forfootskip}%
  {3\baselineskip}
\newlength{\textblockheight}            % Calculate height of text block to leave room
\setlength{\textblockheight}{9.0in}     %   for footers, keeping page numbers outside
\addtolength{\textblockheight}%         %   the 1in vertical margins
  {-\forfootskip}
\settypeblocksize{\textblockheight}%    % Calculated by 1.0in vertical margins and
  {*}{*}                                %   letting margins set the width of the typeblock
\setulmargins{1.0in}{*}{*}              % Set upper margin (\uppermargin, not \topmargin);
                                        %   calculate the bottom margin
\setlrmarginsandblock{1.25in}{1.25in}{*}%~Set margins and calculate width of typeblock
\setheaderspaces{*}{0.5\baselineskip}{*}% Arguments: '\headdrop', '\headsep', and/or ratio
                                        %   Note: This is only used in the list of
                                        %   contents sections
\setheadfoot{\baselineskip}%            % Set '\headheight' and '\footskip'
  {\forfootskip}
\checkandfixthelayout                   % Required by memoir package after setting layout
\settypeoutlayoutunit{in}               % Write layout dimensions to log file in inches

%%%%%%%%%%%%%%%%%%%%%%%%%%%%%%%%%%%%%%%
% Fonts
%%%%%%%%%%%%%%%%%%%%%%%%%%%%%%%%%%%%%%%
\usepackage[T1]{fontenc}                % Standard option to handle, e.g., accented
                                        %   characters like 'ö' better
\usepackage{amssymb,mathtools}          % For AMS-LaTeX, see here for more:
                                        %     http://www.ams.org/publications/authors/tex/amslatex
                                        % ('mathtools' loads and extends 'amsmath')
\usepackage{ifxetex,ifluatex}           % Can check if XeTeX or LuaTeX was used to typeset
\usepackage{fixltx2e}                   % Provides \textsubscript
\IfFileExists{upquote.sty}%             % Use upquote if available, for
  {\usepackage{upquote}}{}              %   straight quotes in verbatim environments

% Load fonts depending on the
%   typesetting engine
\ifnum 0\ifxetex 1\fi\ifluatex 1\fi=0   % If pdftex
  \usepackage[utf8]{inputenc}           %   'utf8' should match the encoding of this file
                                        %
                                        %~Set up your font in pdftex here
                                        %
\else 									                % If xetex or luatex
  \ifxetex                              % If xetex
    \usepackage{mathspec}               % Matches non-math open-type font to math
                                        %   open-type font (use 'mathspec' if you want to
                                        %   write math in unicode)
    \usepackage{xunicode}               % Convert LaTeX character macros to unicode
  \else                                 % If luatex\usepackage{fontspec}
    \usepackage{fontspec}               % Use fontspec for (open type) font selection
  \fi
  \defaultfontfeatures{Mapping=tex-text,% Font spec setting
    Scale=MatchLowercase}
  \newcommand*{\euro}{€}
  \setmainfont{Garamond}                %<Set the main font; make sure the font is correct
                                        %   for the font size (See ASU Style Guide)
%  \setmathfont(Digits,Latin,Greek)%     %~Uncomment two lines to set a font for math%
%    {MATHFONT}
\fi

%%%%%%%%%%%%%%%%%%%%%%%%%%%%%%%%%%%%%%%
% Line spacing
%%%%%%%%%%%%%%%%%%%%%%%%%%%%%%%%%%%%%%%
\DoubleSpacing                          % True double spacing
\BeforeBeginEnvironment{quote}          % Memoir leaves most special material
  {\SingleSpacing}                      %   single spaced, but makes block quotes
\AfterEndEnvironment{quote}%            %   double-spaced; fix to follow ASU style guide
  {\vspace{-\baselineskip} %
  \DoubleSpacing}
\BeforeBeginEnvironment{quotation}%
  {\SingleSpacing}
\AfterEndEnvironment{quotation}%
  {\vspace{-\baselineskip} %
  \DoubleSpacing}

\setlength{\footnotesep}{\baselineskip} % Double space *between* footnotes
\renewcommand*{\footnoterule}{%         % Redefine footnoterule so that initial footnote
  \kern-3pt%                            %   still appears right under the rule (changing
  \hrule width 0.4\columnwidth          %   \footnotesep also changes the space between the
  \kern 2.6pt                           %   rule and the first footnote
  \vspace{-0.5\baselineskip}            % (Here is the vertical space adjustment)
  }

\usepackage{enumitem}                   % Control spacing in enumerate environment
\setlist{noitemsep}                     % Remove extra vertical spacing between items in lists
                                        % \setlist{nosep} to leave no space around whole list

%%%%%%%%%%%%%%%%%%%%%%%%%%%%%%%%%%%%%%%
% Page numbering
%%%%%%%%%%%%%%%%%%%%%%%%%%%%%%%%%%%%%%%
\makepagestyle{ASU}
  \makeevenfoot{ASU}{}{\thepage}{}
  \makeoddfoot{ASU}{}{\thepage}{}

%%%%%%%%%%%%%%%%%%%%%%%%%%%%%%%%%%%%%%%
% Title page, formatting
%%%%%%%%%%%%%%%%%%%%%%%%%%%%%%%%%%%%%%%
\newlength{\savedfootskip}
\setlength{\savedfootskip}{\footskip}
\newcommand{\titlepagesetup}{%          % Page layout for title page
  \changepage%                          % Adjustment to page dimensions:
    {\savedfootskip}%                   %   text height
    {}%                                 %   text width
    {}%                                 %   even-side margin
    {}%                                 %   odd-side margin
    {}%                                 %   column sep.
    {}%                                 %   topmargin
    {}%                                 %   headheight
    {}%                                 %   headsep
    {-\savedfootskip}%                  %   footskip
}

\newcommand{\closetitlepagesetup}{%     % Undo set up for title page
  \changepage{-\savedfootskip}{}{}{}{}%
    {}{}{}{\savedfootskip}%
}

\makeatletter                           % Do not modify this section; Enter info above
\newcommand*{\titlepageASU}{
  \titlepagesetup
  \clearpage
  \begin{center}
  \SingleSpacing
  \thispagestyle{empty}
    \renewcommand*{\do}[1]{##1 \\[\baselineskip]}
    \dolistloop{\titlelines}
    by \\[\baselineskip]
    \Author \\[4\baselineskip]
    A \documentname~Presented in Partial Fulfillment \\
    of the Requirements for the Degree \\
    \degreename \\
    \vfill                              % Vertically center the portion below
    Approved \defdate~by the \\
    Graduate Supervisory Committee: \\[\baselineskip]
    \renewcommand*{\do}[1]{##1, \chairlabel \\}
    \dolistloop{\committeechair}
    \renewcommand*{\do}[1]{##1 \\}
    \dolistloop{\committeemember}
    \vfill                              % Vertically center the portion above
    ARIZONA STATE UNIVERSITY \\[\baselineskip]
    \graddate
  \end{center}
  \clearpage
  \closetitlepagesetup
}
\makeatother

%%%%%%%%%%%%%%%%%%%%%%%%%%%%%%%%%%%%%%%
% Heading styles
%%%%%%%%%%%%%%%%%%%%%%%%%%%%%%%%%%%%%%%
% Note: memoir also has \book and \part commands; do not use these
\makechapterstyle{ASU}{%                % Define chapter heading style
  \renewcommand*{\chapterheadstart}{}   % Chapter title flush with top margin
  \renewcommand*{\chapnamefont}%        % Set font for 'Chapter' or 'Appendix'
    {\normalfont}
  \renewcommand*{\chapnumfont}%         % Set font for number in chapter headings
    {\normalfont}
  \renewcommand*{\afterchapternum}%     % Insert a double line break after
    {\\[\baselineskip]}                 %   chapter number
  \renewcommand*{\chaptitlefont}%       % Set font for chapter title name
    {\normalfont}
  \setlength{\afterchapskip}{0pt}       % Set vertical space between chapter title and
                                        %   first paragraph; equivalent to one line break
                                        %   (vertical space = \afterchapskip + \baselineskip)
                                        % Note: This \afterchapskip value is only used in
                                        %   front matter
  \renewcommand*{\printchapternum}{%    % Center justify chapter number
    \centering \chapnumfont %
    \thechapter}
  \renewcommand*{\printchaptertitle}[1]%% Center justify
    {\expandafter\centering %           %   \MakeUppercase has issues; see here for some
    \expandafter\chaptitlefont %        %   details: https://tex.stackexchange.com/questions/35680/uppercase-in-newcommand
    \expandafter\MakeUppercase %        %   Accented characters and some fonts may not
    \expandafter{##1}}                  %   uppercase correctly; if that happens, just
                                        %   type the chapter title in uppercase
}

\setsecnumdepth{all}                    %~Enter the levels that you want to have numbered
                                        %   (Default is to number all [5 levels deep].)

\newcommand{\divisionbeforeskip}%       % Create default formatting for headings
  {\baselineskip}
\newcommand{\divisionindent}%
  {0.5em}
\newcommand{\divisionfont}{\normalfont} % Font must be \normalfont
\newcommand{\divisionafterskip}%
  {\baselineskip}

\setbeforesecskip{\divisionbeforeskip}  % Apply default formatting to all heading levels
\setsecindent{\divisionindent}          % Note: If you change \setsecnumdepth above, you
\setsecheadstyle{\divisionfont}         %   will need to set the indent for all lower
\setaftersecskip{\divisionafterskip}    %   levels to '0pt'; otherwise, they will be
                                        %   preceded by unnecessary space
\setbeforesubsecskip{\divisionbeforeskip}
\setsubsecindent{\divisionindent}
\setsubsecheadstyle{\divisionfont}
\setaftersubsecskip{\divisionafterskip}

\setbeforesubsubsecskip{\divisionbeforeskip}
\setsubsubsecindent{\divisionindent}
\setsubsubsecheadstyle{\divisionfont}
\setaftersubsubsecskip{\divisionafterskip}

\setbeforeparaskip{\divisionbeforeskip}
\setparaindent{\divisionindent}
\setparaheadstyle{\divisionfont}
\setafterparaskip{\divisionafterskip}

\setbeforesubparaskip{\divisionbeforeskip}
\setsubparaindent{\divisionindent}
\setsubparaheadstyle{\divisionfont}
\setaftersubparaskip{\divisionafterskip}

%%%%%%%%%%%%%%%%%%%%%%%%%%%%%%%%%%%%%%%
% Paragraph formatting
%%%%%%%%%%%%%%%%%%%%%%%%%%%%%%%%%%%%%%%
%\sloppybottom                          % Reduce the chances of widows
\raggedbottom                           % Loosens vertical spacing requirements, so
                                        %   \sloppybottom doesn't make pages look bad;
                                        %   it also prevents large gaps in the middle of
                                        %   pages and pushes them to the bottom of pages
\indentafterchapter                     % Overrides the default which is not to indent
                                        %   the first paragraph in a chapter, but it
                                        %   looks odd in some places to not indent
                                        %   paragraphs

%%% List Titles %%%
\renewcommand{\contentsname}%           % Set heading for each list
  {Table of Contents}%                  %   Formatted as chapter headings by default, so
\renewcommand{\listtablename}%          %   no additional heading formatting is needed
  {List of Tables}
\renewcommand{\listfigurename}%
  {List of Figures}

%%% Depth %%%
\settocdepth{subparagraph}              % Include 5 levels deep (all levels) in TOC

%%% Fonts %%%
\makeatletter%
\patchcmd{\l@part}%                     % Patch the command that writes part-level entries
    {\cftpartfont {#1}}%                %   to the table of contents, so they are in
    {\normalfont \texorpdfstring{%      %   'normalfont' and uppercase
      \uppercase{#1}}{{#1}} }%
    {\typeout{Success: Patch %
      'l@part' to uppercase %
      part-level headings in the %
      table of contents.}}%
    {\typeout{Fail: Patch %
      'l@part' to uppercase %
      part-level headings in the %
      table of contents.}}%
\makeatother%

\makeatletter%
\patchcmd{\l@chapapp}%                  % Patch the command that writes chapter-level
    {\cftchapterfont {#1}}%             %   entries to the table of contents, so they are
    {\normalfont \texorpdfstring{%      %   in 'normalfont' and uppercase
      \uppercase{#1}}{{#1}} }%
    {\typeout{Success: Patch %
      'l@chapapp' to uppercase %
      part-level headings in the %
      table of contents.}}%
    {\typeout{Fail: Patch %
      'l@chapapp' to uppercase %
      part-level headings in the %
      table of contents.}}%
\makeatother%

% If not using 'hyperref', use the following commands to adjust 'part' and 'chapter'
%   level headings in the TOC
%\renewcommand*{\cftpartfont}%          % Uppercase 'part' and 'chapter' headings
%  {\normalfont\MakeTextUppercase}      % Note: Sending \MakeTextUppercase to the TOC
%\renewcommand*{\cftchapterfont}%       %   conflicts with hyperref and breaks it!
%  {\normalfont\MakeTextUppercase}%

\usepackage{titlecaps}                  % Set up headline style for captions in the
                                        %   lists of tables and figures
                                        % Note: ASU style guide does not provide
                                        %   comprehensive guidelines for headlines, so
                                        %   Chicago style for headline style is used
                                        % Note: Last word in title is not explicitly
                                        %   capitalized; in general, these settings are
                                        %   broadly correct, but captions should be
                                        %   reviewed to ensure they are being capitalized
                                        %   properly
\Resetlcwords
\Addlcwords{a an the}                   % Leave articles lowercase
\Addlcwords{and but for or nor}         % Leave conjunctions lowercase
\Addlcwords{aboard about above across % % Leave all prepositions lowercase
  after against along amid among anti % %   (This is a [non-exhaustive] list of common
  around as at before behind below %    %   one-word prepositions)
  beneath beside besides between %
  beyond but by concerning considering %
  despite down during except excepting %
  excluding following for from in %
  inside into like minus near of off %
  on onto opposite outside over past %
  per plus regarding round save since %
  than through to toward towards under %
  underneath unlike until up upon %
  versus vs via with within without}
\Addlcwords{ according\space{to} %      % Leave two-word conjunctions lowercase
  ahead\space{of} apart\space{from} %   %   (This is a [non-exhaustive] list of common
  as\space{for} as\space{of} %          %   two-word prepositions.)
  as\space{per} as\space{regards} %
  aside\space{from} astern\space{of} %
  back\space{to} because\space{of} %
  close\space{to} due\space{to} %
  except\space{for} far\space{from} %
  in\space{to} inside\space{of} %
  instead\space{of} left\space{of} %
  near\space{to} next\space{to} %
  on\space{to} opposite\space{of} %
  opposite\space{to} out\space{from} %
  out\space{of} outside\space{of} %
  owing\space{to} prior\space{to} %
  pursuant\space{to} rather\space{than} %
  regardless\space{of} right\space{of} %
  subsequent\space{to} such\space{as} %
  thanks\space{to} that\space{of} %
  up\space{to}}

\renewcommand{\cfttableaftersnumb}%     % Put table captions in List of Tables in title
  {\titlecap}%                          %   case
\renewcommand{\cftfigureaftersnumb}%    % Put table captions in List of Figures in title
  {\titlecap}%                          %   case

\renewcommand*{\cftpartpagefont}%       % Use normal font for all page numbers
  {\normalfont}
\renewcommand*{\cftchapterpagefont}%
  {\normalfont}
\renewcommand*{\cftsectionpagefont}%
  {\normalfont}
\renewcommand*{\cftsubsectionpagefont}%
  {\normalfont}
\renewcommand*{\cftsubsubsectionpagefont}%
  {\normalfont}
\renewcommand*{\cftsubsubsectionpagefont}%
  {\normalfont}
\renewcommand*{\cftparagraphpagefont}%
  {\normalfont}
\renewcommand*{\cftsubparagraphpagefont}%
  {\normalfont}
\renewcommand*{\cftfigurepagefont}%
  {\normalfont}
\renewcommand*{\cfttablepagefont}%
  {\normalfont}

\cftpagenumbersoff{part}                % Turn off page numbers for 'part's, which are
                                        %   actually serving as headings within the TOC

%%% Vertical Space %%%
\setlength{\cftbeforepartskip}{0pt}     % Remove all additional vertical spacing so TOC
\setlength{\cftbeforechapterskip}{0pt}  %   is double spaced uniformly
\setlength{\cftbeforesectionskip}{0pt}
\setlength{\cftbeforesubsectionskip}{0pt}
\setlength{\cftbeforesubsubsectionskip}{0pt}
\setlength{\cftbeforeparagraphskip}{0pt}
\setlength{\cftbeforesubparagraphskip}{0pt}
\setlength{\cftbeforefigureskip}{0pt}
\setlength{\cftbeforetableskip}{0pt}

\renewcommand{\insertchapterspace}{%    % By default, extra vertical space (10pt) is
  \addtocontents{lof}%                  %   inserted between tables and figures from
    {\protect\addvspace{0pt}}%          %   different chapters; remove this extra space.
  \addtocontents{lot}%
    {\protect\addvspace{0pt}}%
}

%%% Horizontal Space %%%
\newlength{\levelindentincrement}       % Set indent to increase by the same amount for
\setlength{\levelindentincrement}{2em}  %   each level in the TOC; don't adjust figure
\newlength{\levelindent}                %   or table indents
\setlength{\levelindent}%
  {\levelindentincrement}
\setlength{\cftchapterindent}%
  {\levelindent}
\addtolength{\levelindent}%
  {\levelindentincrement}
\setlength{\cftsectionindent}%
  {\levelindent}
\addtolength{\levelindent}%
  {\levelindentincrement}
\setlength{\cftsubsectionindent}%
  {\levelindent}
\addtolength{\levelindent}%
  {\levelindentincrement}
\setlength{\cftsubsubsectionindent}%
  {\levelindent}
\addtolength{\levelindent}%
  {\levelindentincrement}
\setlength{\cftparagraphindent}%
  {\levelindent}
\addtolength{\levelindent}%
  {\levelindentincrement}
\setlength{\cftsubparagraphindent}%
  {\levelindent}
\addtolength{\levelindent}%
  {\levelindentincrement}

\setlength{\cftchapternumwidth}%        % Decrease space between number and heading for
  {0.85\cftchapternumwidth}             %   all heading levels
\setlength{\cftsectionnumwidth}%
  {0.85\cftsectionnumwidth}
\setlength{\cftsubsectionnumwidth}%
  {0.85\cftsubsectionnumwidth}
\setlength{\cftsubsubsectionnumwidth}%
  {0.85\cftsubsubsectionnumwidth}
\setlength{\cftparagraphnumwidth}%
  {0.85\cftparagraphnumwidth}
\setlength{\cftsubparagraphnumwidth}%
  {0.85\cftsubparagraphnumwidth}
\setlength{\cftfigurenumwidth}%         % Figure has the same 'level' as 'chapter' in the
  {\cftchapternumwidth}                 %   figure list, so make the number spacing the
                                        %   same as for chapters
\setlength{\cfttablenumwidth}%          % Table has the same 'level' as 'chapter' in the
  {\cftchapternumwidth}                 %   table list, so make the number spacing the
                                        %   same as for chapters

%%% Leaders/dots %%%
\renewcommand*{\cftdotsep}{1.7}         % Set distance between dots for all heading levels
\renewcommand*{\cftchapterleader}%      % Turn on dots for 'chapter' level
  {\normalfont\cftdotfill{\cftdotsep}}
\makeatletter                           % Bring leader dots over to page number (no gap)
  \renewcommand{\@pnumwidth}{1.55em}    %~Manually adjust
  \renewcommand{\@tocrmarg}{2.55em}
\makeatother

\renewcommand{\cfttableaftersnum}{.}    % Period after number in LOT
\renewcommand{\cftfigureaftersnum}{.}   % Period after number in LOF

%%% Printing List Titles and Headers in Content Lists
% Table of Contents (TOC)
\copypagestyle{ASUtoc}{ASU}%            % Page style for regular page in TOC
  \makeevenhead{ASUtoc}%
    {\leftmark}{}{Page}
  \makeoddhead{ASUtoc}%
    {\leftmark}{}{Page}

\copypagestyle{ASUtocFirst}{ASU}%       % Custom page headers for first page of TOC
  \makeevenhead{ASUtocFirst}%           %    (print out the title)
    {}%
    {\printchaptertitle{\contentsname}}%
    {}
  \makeoddhead{ASUtocFirst}%
    {}%
    {\printchaptertitle{\contentsname}}%
    {}

\renewcommand{\tocheadstart}{}%         % Usually content list titles are printed like
                                        %   chapter headings; empty that formatting

\renewcommand{\printtoctitle}[1]{}%     % Don't print TOC title using default method;
                                        %   it will be output in the header

\renewcommand{\aftertoctitle}{%         % On the first page of the TOC, print out the
  \thispagestyle{ASUtocFirst}%          %   TOC title using a custom page style and print
  \hfill Page\par%                      %   the heading for the page below in the regular
  }%                                    %   textbox
                                        % Note: Need '\par' before lists; see here: https://tex.stackexchange.com/questions/49882/yet-another-perhaps-a-missing-item-error

% List of Tables (LOT)
\copypagestyle{ASUlot}{ASU}%            % Page style for regular page in list of tables
  \makeevenhead{ASUlot}{Table}{}{Page}
  \makeoddhead{ASUlot}{Table}{}{Page}

\copypagestyle{ASUlotFirst}{ASU}%       % Custom page headers for first page of list of
  \makeevenhead{ASUlotFirst}%           %   tables (print out the title)
    {}%
    {\printchaptertitle{\listtablename}}%
    {}
  \makeoddhead{ASUlotFirst}%
    {}%
    {\printchaptertitle{\listtablename}}%
    {}

\renewcommand{\lotheadstart}{}%         % Usually content list titles are printed like
                                        %   chapter headings; empty that formatting;

\renewcommand{\printlottitle}[1]{}%     % Don't print LOT title using default method;
                                        %   it will be output in the header

\renewcommand{\afterlottitle}{%         % On the first page of the list of tables, print
  \thispagestyle{ASUlotFirst}%          %   out the title using a custom page style and
  Table\hfill Page\par}%                %   print heading below in regular textbox

% List of Figures (LOF)
\copypagestyle{ASUlof}{ASU}
  \makeevenhead{ASUlof}{Figure}{}{Page}
  \makeoddhead{ASUlof}{Figure}{}{Page}

\copypagestyle{ASUlofFirst}{ASU}%       % Custom page headers for first page of list of
  \makeevenhead{ASUlofFirst}%           %   figures (print out the title)
    {}%
    {\printchaptertitle{\listfigurename}}%
    {}
  \makeoddhead{ASUlofFirst}%
    {}%
    {\printchaptertitle{\listfigurename}}%
    {}

\renewcommand{\lofheadstart}{}%         % Usually content list titles are printed like
                                        %   chapter headings; empty that formatting

\renewcommand{\printloftitle}[1]{}%     % Don't print LOF title using default method;
                                        %   it will be output in the header

\renewcommand{\afterloftitle}{%         % On the first page of the list of figures, print
  \thispagestyle{ASUlofFirst}%          %   out the title using a custom page style and
  Figure\hfill Page\par}                %   print heading below in regular textbox

%%% Page layout (dimensions) for Contents Lists
\newlength{\verticalpush}               % Set up to change page dimensions for the table
                                        %   of contents
                                        % Push everything down so all the content is still
                                        %   1in from the top of the page, including the
                                        %   header, so the header is available for titles
                                        %   on the first page of contents lists and then
                                        %   the headings on subsequent pages
\setlength{\verticalpush}%              % Calculate difference between \headdrop and the
  {1.0in - \headdrop}                   %   total upper margin (1in), so you can push
                                        %   the top of the header down into the textbox

\newcommand{\contentslistsetup}{%       % Set up for contents lists
  \changepage%                          % Adjustment to page dimensions:
    {-\baselineskip}%                   %   text height
    {}%                                 %   text width
    {}%                                 %   even-side margin
    {}%                                 %   odd-side margin
    {}%                                 %   column sep.
    {\verticalpush}%                    %   topmargin
    {}%                                 %   headheight
    {}%                                 %   headsep
    {-\verticalpush+\baselineskip}%     %   footskip
}

\newcommand{\closecontentslistsetup}{%  % Undo set up for contents lists
  \changepage{\baselineskip}{}{}{}{}%
    {-\verticalpush}{}{}{\verticalpush-\baselineskip}%
}

% Content lists can also be output directly. If the following command were used, all the
%   headings would have to be output manually (i.e., can't rely on any memoir macros for
%   formatting or setting in contents lists headings and lists). It would be best to
%   create a custom macro, such as '\customtoc', to output headings and content lists
%   following the style guide.
%
% \makeatletter
%   \@starttoc{toc}
% \makeatother

% These pages partly explain why it's difficult to use 'afterpage' to change page layout
%   settings (essentially, it's because everything inside \afterpage has a local scope).
%   If it were possible to use 'afterpage' in that way, the content lists would  be
%   easier to format. A new page layout could be called after the first page of each
%   content  list. Instead, use page marks to get the layout required by the style guide.
% https://tex.stackexchange.com/questions/97126/attempts-to-manually-change-linewidth-ignored-by-latex
% https://tex.stackexchange.com/questions/85729/page-styles-only-work-for-thispagestyle-under-afterpage

%%%%%%%%%%%%%%%%%%%%%%%%%%%%%%%%%%%%%%%
% Footnotes and Endnotes
%%%%%%%%%%%%%%%%%%%%%%%%%%%%%%%%%%%%%%%
\usepackage{chngcntr}                   % Modify counters (e.g., for figures, footnotes)
\counterwithout*{footnote}{chapter}     % Make footnote numbering continuous throughout

\providetoggle{useendnotes}
\settoggle{useendnotes}{true}           %<Set to 'true' if you want to use endnotes
\iftoggle{useendnotes}{%                % Use the command \pagenote to create endnotes
                                        %   in the running text. They will be collected
                                        %   and printed in a 'Notes' section at the end
                                        %   of the document

  \makepagenote                         % Required in preamble if using endnotes
  \continuousnotenums                   % Numbering does *not* reset after each chapter
  \renewcommand*{\pagenotesubhead}[3]{} % No subheads inside note list (default is to
                                        %   divide them by chapter)
  \renewcommand*{\notenuminnotes}[1]%   % Remove extra space between note number and note
    {\normalfont #1.}                   %   text
  \renewcommand{\postnoteinnotes}%      % Double space *between* notes
    {\par\vspace{\baselineskip}}
}{}                                     % Do nothing here if not using endnotes

%%%%%%%%%%%%%%%%%%%%%%%%%%%%%%%%%%%%%%%
% Bibliography
%%%%%%%%%%%%%%%%%%%%%%%%%%%%%%%%%%%%%%%
\newcommand{\bibfilename}{sample_library}%<Enter the name of the *.bib file containing the
                                        %   reference information for sources cited in
                                        %   the text. God help you if you're doing
                                        %   citations manually.
\newcommand{\bibheading}{References}    %<Enter the heading for the references section:
                                        %   'References', 'Works Cited', or 'Bibliography'

\providetoggle{usebiblatex}             % True = a biblatex package is being used;
                                        %   False = 'natbib' is being used
\settoggle{usebiblatex}{true}           %~Set to 'false' to use 'natbib' intead of
                                        %   biblatex; I strongly recommend using biblatex
                                        %   because natbib is rather old and will break
                                        %   for innocuous things like underscores in URLs
\iftoggle{usebiblatex}{%                % Settings for citation package
%                                       % Settings for 'biblatex' or a version of
%                                       %   'biblatex'
  \usepackage[authordate,%
              backend=biber,%           % Recommend to use 'biber' instead of 'bibtex'
              doi=only,%                % Avoid printing URLs
              isbn=false]%              % Don't print ISBN numbers
              {biblatex-chicago}        %~Other possibilities include: 'biblatex',
                                        %   'biblatex-apa', and 'biblatex-mla'
  \bibliography{\bibfilename}
  \setlength{\bibitemsep}%              % Set vertical distance between
    {0.5\baselineskip}%                 %   bibliography entries
  \setcounter{biburlnumpenalty}{9000}   % Break URLs in bibliography across lines
  \setcounter{biburlucpenalty}{9000}
  \setcounter{biburllcpenalty}{9000}

  \usepackage[style=american,%          % Settings for quotation marks; load after
    english=american]{csquotes}%        %   'inputenc'; only use with biblatex; throws
  \MakeOuterQuote{"}%                   %   error when used with natbib
}{%                                     % Settings for 'natbib'
  \usepackage{natbib}%
  \newcommand{\natbibstyle}{asudis}%    %~Enter the name of the *.bst file to use to
                                        %   format citations with natbib. Default is
                                        %   'asudis'. I do not know where 'asudis' came
                                        %   from, but apparently it formats citations
                                        %   correctly because it was included with the
                                        %   previous LaTeX template.
}

%%%%%%%%%%%%%%%%%%%%%%%%%%%%%%%%%%%%%%%
% Tables and figures
%%%%%%%%%%%%%%%%%%%%%%%%%%%%%%%%%%%%%%%
\captiondelim{. }                       %~Use period (.) after caption number instead of
                                        %   colon (:). Change according to style guide.
\captionstyle[\raggedright]%            % Set justifcation for [one line captions]
  {\raggedright}                        %   and {multiple line captions}
\setlength{\belowcaptionskip}{0pt}      % Bring caption down closer to figure/table
\makeatletter                           % Consecutive numbering throughout
  \counterwithout{figure}{chapter}      %   (including back matter)
  \counterwithout{table}{chapter}
  \renewcommand\@memfront@floats{}
  \renewcommand\@memmain@floats{}
  \renewcommand\@memback@floats{}
\makeatletter

\newcommand{\macrocapwrap}[1]{%         % Use this macro to place other macros inside
  {\bgroup\bgroup{{#1}}\egroup\egroup}% %   captions, e.g., '\macrocapwrap{\ref{figure1}}'
}%                                      % Note: Necessary due to the 'titlecaps' package
                                        %   which modifies fcs of captions

%%% Tables %%%
%
% Note: 'memoir' natively supports commands from the following table-related packages:
%   tabularx, ccaption, booktabs.
% Everyone has particular ideas about how tables should look, so you may need to
%   load additional packages and modify the code below to get tables (and figures) to
%   look the way you want them to.
\setfloatadjustment{table}{\raggedright}% Left justify material inside table floats
\usepackage{tabu}                       % 'tabu' is an excellent table package; it can
                                        %   automatically size column widths and has a
                                        %   lot of customizations that other packages do
                                        %   not. It also has a 'longtabu' environment that
                                        %   emulates 'longtable' with additional features
                                        %   from the 'tabu' package. If you don't want
                                        %   to use it, you can comment this line out.
\BeforeBeginEnvironment{table}%         % Single space inside table environment
  {\SingleSpacing}
\AfterEndEnvironment{table}
  {\DoubleSpacing}

%%% Figures %%%
\setfloatadjustment{figure}%            % Left justify material inside figure floats
  {\raggedright}
\BeforeBeginEnvironment{figure}%        % Single space inside figure environment
  {\SingleSpacing}
\AfterEndEnvironment{figure}
  {\DoubleSpacing}

\makeatletter                           % Define custom macro called '\maxwidth{}' that
  \def\maxwidth#1{%                     %   allows you to specify the maximum width of an
    \ifdim%                             %   imported image. See below for an example.
      \Gin@nat@width>#1 #1%             %
    \else%                              % Source: http://tex.stackexchange.com/questions/86350/includegraphics-maximum-width
      \Gin@nat@width%
    \fi}
\makeatother
%
% Example \maxwidth:
%
%   \includegraphics[width=\maxwidth]{\textwidth}]{image.pdf}
%
% Note: This will keep an image inside the horizontal margins assuming the image starts
%   on the right margin (i.e., no horizontal space before the image).

%%%%%%%%%%%%%%%%%%%%%%%%%%%%%%%%%%%%%%%
% Hyperref settings
%%%%%%%%%%%%%%%%%%%%%%%%%%%%%%%%%%%%%%%

%%% URL Settings %%%
\PassOptionsToPackage{hyphens}{url}
\usepackage[breaklinks=true]{hyperref}  % 'hyperref' should be loaded at the end of the
                                        %   preamble; Note: the uppercasing commands used
                                        %   throughout the preamble can conflict with it,
                                        %   especially when non-standard fonts or
                                        %   different file encodings are used
\urlstyle{same}                         % Set URLs in the same font as regular text

\tolerance 1414                         % Help URLs from entering margins
\hbadness 1414                          %   Source: https://tex.stackexchange.com/questions/3033/forcing-linebreaks-in-url
\emergencystretch 1.5em
\hfuzz 0.3pt
\widowpenalty=10000
\vfuzz \hfuzz

%%% Create metadata strings
\usepackage{hyperxmp}                   % For metadata
\renewcommand*{\do}[1]{#1\ }%           % Build title string to output to pdf document
\newcommand*{\onelinetitle}{%
  \dolistloop{\titlelines}%
}
\edef\theonelinetitle%
  {\onelinetitle}

\renewcommand*{\do}[1]{{#1}\ }%          % Build keyword string to output to pdf document
\newcommand*{\pdfkeywordsstring}{%
  \dolistloop{\keywords}%
}
\edef\thepdfkeywordsstring%
  {\pdfkeywordsstring}

\newcommand*{\pdfcopyrightstring}%      % Build copyright message string
  {Copyright \copyright\space\gradyear\ by \Author.%
  {\space}All rights reserved.}

\ifpdf                                  % Build pdf creator string (for pdfTeX)
  \makeatletter
  \def\extractpdftexversion#1-#2-#3 #4%
    \@nil{#3}
  \edef\pdfcreator{pdfTeX \expandafter%
    \extractpdftexversion\pdftexbanner\@nil}
  \makeatother
\fi
\ifxetex                                % Build pdf creator string (for XeTeX)
  \edef\pdfcreator{XeTeX %
    \the\XeTeXversion\XeTeXrevision}
\fi

\edef\pdfsummary{%                      % Build pdf summary
  A \documentname Presented in\space
  Partial Fulfillment of the\space
  Requirements for a \degreename\space
  from Arizona State University}

%%% Enter metadata and other settings
\hypersetup{                            % Set pdf metadata
  pdftitle={\theonelinetitle},          % Title
  pdfauthor={\Author},                  % Author
  pdfcreator={\pdfcreator},             % Enter the TeX writer for good documentation
 %pdfproducer={},                       % Let 'pdfproducer' be filled automatically
  pdfsubject={\pdfsummary},             % Subject of the document
  pdfkeywords=\thepdfkeywordsstring,    % List of keywords
  hidelinks={true},                     % Links look like regular text (no colors, boxes)
  breaklinks={true},                    % Allow links to break across lines
}
\ifxetex                                % If processing with XeTeX
  \hypersetup{unicode=true}             % Must use 'true' in XeTeX
\else
  \hypersetup{unicode=true}             % Default is to use 'true' otherwise, as well
\fi
\ifpdf                                  % Copyright message; probably only works in pdfTeX
  \hypersetup{
    pdfcopyright={\pdfcopyrightstring},
    pdfinfo={%
      Copyright=\pdfcopyrightstring%
    }%
  }
\fi

\usepackage%
  [numbered,%                           % Include numbers of sections in bookmarks
  open%                                 % Bookmark tree already expanded when PDF opened
  ]%
  {bookmark}
\bookmark[page=1,rellevel=0,%           % Create bookmark of title page at root level
  keeplevel=true]{Title Page}
\preto{\tableofcontents}{%              % Create bookmark for TOC
  \hypertarget{tocpage}{}%
  \bookmark[dest=tocpage,rellevel=0,%
    keeplevel=true]{\contentsname}%
}

%%%%%%%%%%%%%%%%%%%%%%%%%%%%%%%%%%%%%%%
% Copyright page
%%%%%%%%%%%%%%%%%%%%%%%%%%%%%%%%%%%%%%%
\newcommand{\copyrightpageASU}{%        % Create copyright page
  \thispagestyle{empty}
  \titlepagesetup
  ~\\ \vfill
    \begin{center}
      \copyright\gradyear\space%
      \Author\\%
      All Rights Reserved%
    \end{center}%
  \clearpage%
  \closetitlepagesetup
}

%%%%%%%%%%%%%%%%%%%%%%%%%%%%%%%%%%%%%%%
% Sample settings
%%%%%%%%%%%%%%%%%%%%%%%%%%%%%%%%%%%%%%%
\providetoggle{sample}                  % True = demonstration of template
\settoggle{sample}{true}
\iftoggle{sample}{%
  \newcounter{tablecounter}
  \setcounter{tablecounter}{1}
  \newcounter{figurecounter}
  \setcounter{figurecounter}{1}
}{%
}

%%%%%%%%%%%%%%%%%%%%%%%%%%%%%%%%%%%%%%%
% Debugging Help
%%%%%%%%%%%%%%%%%%%%%%%%%%%%%%%%%%%%%%%
\usepackage{lipsum}                     % Outputs dummy text
\usepackage{subfiles} % Best loaded last in the preamble

%%%%%%%%%%%%%%%%%%%%%%%%%%%%%%%%%%%%%%%%%%%%%%%%%%%%%%%%%%%%%%%%%%%%%%%%%%%%%%
% Document
%%%%%%%%%%%%%%%%%%%%%%%%%%%%%%%%%%%%%%%%%%%%%%%%%%%%%%%%%%%%%%%%%%%%%%%%%%%%%%
\begin{document}


%%%%%%%%%%%%%%%%%%%%%%%%%%%%%%%%%%%%%%%
% Title page
%%%%%%%%%%%%%%%%%%%%%%%%%%%%%%%%%%%%%%%
\titlepageASU

%%%%%%%%%%%%%%%%%%%%%%%%%%%%%%%%%%%%%%%
% Copyright page
%%%%%%%%%%%%%%%%%%%%%%%%%%%%%%%%%%%%%%%
%\copyrightpageASU                       %~If you don't want to have a copyright page,
                                        %   comment out this line

%%%%%%%%%%%%%%%%%%%%%%%%%%%%%%%%%%%%%%%
% Front matter
%%%%%%%%%%%%%%%%%%%%%%%%%%%%%%%%%%%%%%%
\chapterstyle{ASU}
\pagestyle{ASU}
\frontmatter

\chapter*{Abstract}                     % Abstract is required



%\input{template/sample_abstract}                 %<Enter the name of the .tex file containing your
                                        %   your abstract or omit this line and type in
                                        %   your abstract here.

%\chapter*{Dedication}                   %~Dedication is optional
%\clearpage                             %~If you don't wish to display the heading
                                        %   'Dedication', comment out the previous line
                                        %   and use this one instead.
%\leavevmode\vfill
I dedicate this theis to my parents who put a value on higher education.
\vfill               %<Enter the name of the .tex file containing your
                                        %   your dedication or omit this line and type in
                                        %   your dedication here.

\chapter*{Acknowledgments}              %~Acknowledgments are optional
\input{template/sample_acknowledgments}          %<Enter the name of the .tex file containing your
                                        %   acknowledgments or omit this line and type in
                                        %   your acknowledgments here.

\iftoggle{usemicrotype}                 % If 'microtype' is in use, turn off protrusion
  {\microtypesetup{protrusion=false}}%  %   for TOC
  {}
\clearpage                              % Output table of contents on a new page
\contentslistsetup                      % Change page layout for contents lists

\pagestyle{ASUtoc}
\tableofcontents*                       % Starred version leaves TOC heading out of TOC
\addtocontents{toc}%                    % List of ... needs to be on left margin, but
  {\setlength{\cftchapterindent}%       %   they inherit 'chapter' formatting, so override
    {0em}%
  }

\clearpage
\pagestyle{ASUlot}
\listoftables                           % List of Tables should appear in TOC, so use
                                        %   unstarred version of \listoftables
\clearpage
\pagestyle{ASUlof}
\listoffigures                          % List of Figures should appear in TOC, so use
                                        %   unstarred version of \listoffigures

\phantomsection                         % \phantomsection is needed before using
                                        %   \addtocontents when it contains certain macros
                                        %   when also using 'hyperref' package
\addtocontents{toc}%                    % Undo manual override above for chapter indent,
  {\setlength{\cftchapterindent}%       %   so actual chapters in the TOC are indented
    {\levelindentincrement}%            %   correctly
  }
\setlength{\afterchapskip}%             % Set vertical space between chapter title and
  {\baselineskip}                       %   first paragraph; equivalent to two line breaks

\phantomsection
\addcontentsline{toc}{part}{Chapter}    % Add "Chapter" to TOC here at 'part' level
\phantomsection
\addtocontents{toc}%                    % Add this 'mark' to TOC so subsequent pages use
  {\protect\markboth{CHAPTER}{Page}}    %   the "CHAPTER" heading

\iftoggle{usemicrotype}                 % If 'microtype' is in use, turn protrusion back
  {\microtypesetup{protrusion=true}}%   %   on
  {}

\clearpage                              % Note: All these changes have to be above a
                                        %   a '\clearpage' before '\mainmatter'

\pagestyle{ASU}                         % Switch back to regular page style for remainder
                                        %   of the document
\closecontentslistsetup                 % Undo page layout for contents lists

%\chapter{Definitions}                   %~OTHER LISTS (optional)
%\input{definitions}                     %<Enter the name of the .tex file or omit this
                                        %   line and type in here.

%\chapter{Preface}                       %~PREFACE (optional, less than 10 pages)
%\input{preface}                         %<Enter the name of the .tex file or omit this
                                        %   line and type in here.

%%%%%%%%%%%%%%%%%%%%%%%%%%%%%%%%%%%%%%%
% Body
%%%%%%%%%%%%%%%%%%%%%%%%%%%%%%%%%%%%%%%
\mainmatter

\begin{comment}

\subsection{1A}: Optimize reduced neuron models with respect to data from four different neuron types, using two different classes of models. Optimize neuron models to create agreement with electrophysiological measurements.
\subsection{1B:} Validate the optimization approach using analysis of variance in many pre existing models and experimental data together. Test how well optimized models mimic experimental data relative to pre-existing models and data. 

\subsection{2A:} Using large, publicly available datasets for neuron classes, we will characterize region-specific differences and evaluate neuron-type models for agreement with experimental data.
\subsection{2B:} Using models of neuron-types from different cortical regions we will identify the features underlying differences among neuron types and across regions, as well as the biophysical mechanisms that underlie those features. 

\end{comment}






\subsection{Model  Optimization with NeuronUnit}
Some neural properties can’t be easily measured in experiments. These unknown properties hamper modeling accuracy and require parameter fitting. For example, a common approach for approximating unknown ion channel densities is to ‘optimize’ the governing equations to match known waveforms. The process of optimization involves what is known as an ‘inverse’ problem where we intelligently and sparsely search for the ‘optimal’ value of an parameter that satisfies the system of equations. Computational optimization techniques are generally specific to a particular type of problem rather than being generalized. However, several notable algorithms have solved a wide range of problems including non-dominated sort 2 (NSGA2) and stochastic gradient descent (SGD). The popularity of these two algorithms relies on the ability to avoid falsely reporting a local minimum as the most optimal solution. However, of SGD and NSGA2, only NSGA2 is a natural choice for tackling multi-objective optimization problems. Default implementations of SGD are not able to utilize the principle of non-domination as an optimization strategy.\\
\\
NeuronUnit easily converts a quantitative measure of model/data agreement into a useful error signal. A very natural application of this signal is to guide the process of optimization. We have used Neuronunit to guide optimization by taking a flexible model type such as a generalized linear integrate and fire model or the Izhikevich model and constraining the model against relevant experimental data. As an example, NSGA2 was used to optimize models in conjunction with data driven tests based on pooled data from NeuroElectro.org. A variety of compact and fast single compartment models were used to explore model optimization. Figure 4 demonstrates test error at the beginning of the optimization process for models with randomly sampled parameters and the smaller error following optimization. Figure 5 shows the evolution of the error during the optimization process. \\
\\
Optimized neuron models may vary from their neuron counterparts for several reasons. Table 3 shows an example where optimizing the model with respect to the rheobase test comes into conflict with minimizing with respect to input resistance. The solution to the optimization problem consists of two sets of model parameters, which can resolve this conflict differently. Examining the experimental data that these tests were derived from suddenly becomes important. By examining the data, we can see if the rheobase currents and the distributions of input resistance are bi-modal and uniformly distributed. If the data is treated as uni-modal, and the uni-modal mean is used to optimize then the model, then the model is not able to satisfy both constraints simultaneously. In this case, the measurements don’t correspond to neuron data, and the model can’t produce the artificial behavior. When comparing complex data and simple models we find that solutions are better represented using a combination of two optimization solutions.\\
\\
Another potential issue to consider when evaluating the scientific merit of a model is that neurons may have different behaviors under different stimulation paradigms. It might be appropriate to compare modeled behavior against measurements specific to each of two or more distinct modes. In this case, when optimizing single cell models, it’s appropriate to accept a solution set, rather than a single solution. For example, the cerebellar Purkinje cell is sensitive to intricately patterned dendrite input current combinations. Depending on a cell’s recent history of synaptic stimulation, a Purkinje cell may toggle between coincidence detection and integration modes (Ratté, Hong, De Schutter, & Prescott, 2013).  \\
\\
\Section{Ecosystem of Modelling Resources}
The NEURON simulator is a software suite that wraps powerful and fast ordinary differential equation solvers based in the C programming language inside a mixed compiled/interpreted environment targeted at research scientists. NEURON is somewhat analogous to older, analog circuit simulators; however, rather than describing complex resistor-capacitor circuits, NEURON instead solves equations for the time varying membrane potential of multi-compartment models.\\
\\
These multi-compartmental models are based on cables of varying diameters and lengths that represent the morphology of neurons, where these cables support ionic currents in the membranes. These neuronal models can be coupled together into a network, where the electrical state of one neuron has an impact on the state of coupled neurons through synaptic currents. Specifying the system of differential equations representing these neuronal morphologies, ion channels, and synaptic connections is complicated, but NEURON makes multi-compartment neuron simulation efficient, convenient, and achievable. Models expressed in NEURON code are procedural in nature, and the code consists of low-level implementation details. Procedural descriptions of models are difficult to extend and re-use, leading to a need for a declarative model description language. NeuroML has been tasked with describing these with complex network models.\\
\\
Through jNeuroML, the NeuroML project also provides a simple code interface for generating complex simulator code, so that NeuroML models are readily exchanged between different types of simulators. Model interchange permits cross examination of results as a they vary across simulators, and this interchange promotes the movement of models between languages preferred by different modeling communities, reconciling and unifying their models. Because NeuroML is extensible and component based, it incentivizes a “plug-in” environment for including pre-existing model components in models in a different large-scale context.


\section{Significance}
Beyond experimental error, it is common to observe large variations in measurements of a single electrophysiological entity from neurons of the same classification. As an example, consider that measurements of neuron membrane input resistance may be different when recorded from different samples of the same neuron type. This variation is an essential consideration when evaluating the scientific merit of a computational model of a neuron-type. In this work, we propose to perform a large-scale analysis of model against data agreement and model against model agreement to expose the variation in biophysically realistic neuron models and cortical data. By analyzing the variance in data and among models and linking the variation to specific features and mechanisms, we also will better understand the heterogeneity of experimental measurements from a particular neuron-type. Performing a meta-analysis with a large number of models will provide other insights. We will determine whether there is higher variance in modeled electrical properties versus experimental neurophysiological measurements. We will examine whether an extensive collection of cortical models behaves more similarly to each other than to the data and will answer the question: does the space of all existing single cell models accurately represent the variability in experimental data?
Similarly, variation in the behavior of cortical neuronal networks is not well quantified. Tremendous research effort has been consumed producing several high-quality, experimentally informed cortical network models. Before creating another elaborate network model, we will determine whether these pre-existing approaches lead to networks with significantly different dynamic properties. Also, we will create an infrastructure that allows scientists to quantify the similarities and differences among networks and their dynamics – both biological and in silico.\\
\\
Existing data sets are incomplete, consisting of a sparse sampling of cells in the rodent brain. By necessity, models are constrained using these incomplete data sets, leading to compensatory model development that synthesizes missing information. Missing data occurs at multiple levels during network construction including exact neuron to neuron wiring patterns, un-sampled morphologies, unknown synapse activation times, and unknown axon and dendrite synapse locations. Published models should not be regarded as final, but to improve models, it is vital that they are validated against newly-obtained experimental data. The proposed work will facilitate ongoing validation of biophysically realistic models.
Some electrophysiology data are challenging to integrate into existing models. These include data collected from animal species that are not widely used in models such as marmoset, guinea pig, and even humans. Additionally, neuron-type data may come from a widely-used model organism, but the tissue samples may have been extracted from brain tissue in a pathological condition. In practice, open access data is not always useable, as it be derived from multiple species types and brain regions. We will obtain a better understanding of region-dependent differences and species-dependent differences in order to help researchers map models onto a standardized rodent electrophysiological phenotype space.\\


\section{Innovation}
The large-scale meta-analysis described here has not been performed previously. For the first time, a large number of cortical neuron and neuronal network models are available in the standardized NeuroML format. Although the Allen Institute for Brain Science modeling project and the Blue Brain project both rigorously analyzed their single cell models, there has not been an overarching meta-analysis across different cell and network model sources. Similarly, numerous modeling efforts have employed data-driven testing in model development workflows, but all these efforts have been based on non-standard ‘in-house’ model types and execution environments. In contrast, this work proposes to expand a pre-existing standardized model testing space, NeuronUnit, that supports model validation and re-use regardless of the model source. To date various NeuronUnit tests of action potential shape, electrical properties, and single cell morphologies exist; yet these tools are not unified. Some tests of network dynamics also currently exist; however, these tests are not integrated into a unified multiscale workflow. Significantly, a unified workflow would better locate errors in network behavior which are manifest at the network level but are caused by neuron-type models. 

\section{Approach}
Many sophisticated, experimentally informed models of neurons and neuronal networks exist, yet these models are limited to the target neuron class and brain region they were designed to explain. There is a scientific and economical mandate to increase the usability of existing models and data; however, it is unknown whether models developed for a particular cortical area can be re-parameterized to model a different cortical region. A systematic and thorough comparison of neuron model behavior has not been performed, and it is unknown how differently classes of models behave relative to one another and different data. We seek to widen the range of valid model re-purposing by examining relationships across hundreds of cortical neurons, across neuron type models, and by comparing these models to experimental data.\\
\\
Model formats and simulation software: This project will rely on NeuroML, a community-developed format for describing multiscale models in neuroscience supported by over 40 downstream applications, including simulators, databases, and tools for analysis and 

\section{Further Validation of Optimized Models}
Models can continue to undergo validation checks subsequent to optimiztion. One way to do this is to move from a single spiking stimulus regime, to a multispiking regime, and then use different feature extraction libraries using a much larger number of features.  The higher dimensional feature space can be reduced to a low dimensional space, where it is eaasier to discriminate between models and data.


\section{Further Validation of Optimized Models}
Models can continue to undergo validation checks subsequent to optimization. One way to do this is to move from a single spiking stimulus regime, to a multispiking regime, and then use different feature extraction libraries using a much larger number of features. 
%Models can continue to undergo validation checks subsequent to optimization. One way to do this is to move from a single spiking stimulus regime, to a multispiking regime, and then use different feature extraction libraries using a much larger number of features. 
%The higher dimensional feature space can be reduced to a low dimensional space, where it is eaasier to discriminate between models and data.

%Agreement between the optimized.


\chapter*{Results}
\section{PCA and tSNE}

\begin{figure}

%[width=0.5\textwidth]
%\maxwidth=0.9
\includegraphics[width=\maxwidth{\textwidth},scale=0.5]{template/PCA.png}
\includegraphics[width=\maxwidth{\textwidth},scale=0.5]{template/PCA.png}
\caption{Principal Component Analysis}
\label{figure\arabic{figurecounter}}
\legend{Feature Variance distributed though-out a plane, the plane was output from applying dimension reduction to a high dimensional (552 dim) feature Space. The plane is related to two projection vectors that were  point in the directions of maximal variance in the data. Data is comprised by a collection of reduced neuronal models, and real recordings of voltage traces from neurons. Data and models are so different in this space that they are easily discriminated, and they fall into several natural clusters. Random Forests variance explained was later used to identify, dimensions that contributed the most variance.
\emph{T-Distributed Stochastic Nearest Neighbour Spatial Embedding} }

\end{figure}




\begin{figure}
\includegraphics[width=\maxwidth{\textwidth}]{figures/TSNE.png}
\includegraphics[width=\maxwidth{\textwidth}]{template/TSNE.png}
\caption{t distributed stochastic Neigherset Neighbour Spatial Embedding}
\label{figure\arabic{figurecounter}}

\legend{\emph{Source}: \iftoggle{usebiblatex}{\textcite{krishnappa_adult_2012}}{\citet{krishnappa_adult_2012}}}% See:
\legend{\emph{Note}: Here is a note that is especially long to show what happens when it extends to more than one line.}
\end{figure}




\begin{figure}
	\includegraphics[width=\maxwidth{\textwidth}]{figures/TSNE.png}
	\caption{t distributed stochastic Neigherset Neighbour Spatial Embedding}
	\label{figure\arabic{figurecounter}}
\end{figure}

\begin{figure}
	\includegraphics[width=\maxwidth{\textwidth}]{figures/directions_variance.png}
	\caption{t distributed stochastic Neigherset Neighbour Spatial Embedding}
	\label{figure\arabic{figurecounter}}
\end{figure}
\begin{figure}
	\includegraphics[width=\maxwidth{\textwidth}]{figures/results_conductanc_models.png}
	\caption{t distributed stochastic Neigherset Neighbour Spatial Embedding}
	\label{figure\arabic{figurecounter}}
\end{figure}
\begin{figure}
	\includegraphics[width=\maxwidth{\textwidth}]{figures/results_izhi_models.png}
	\caption{t distributed stochastic Neigherset Neighbour Spatial Embedding}
	\label{figure\arabic{figurecounter}}
\end{figure}
\begin{figure}
	\includegraphics[width=\maxwidth{\textwidth}]{figures/normal_distribution.png}
	\caption{t distributed stochastic Neigherset Neighbour Spatial Embedding}
	\label{figure\arabic{figurecounter}}
\end{figure}
\begin{figure}
	\includegraphics[width=\maxwidth{\textwidth}]{figures/results_izhi_waves.png}
	\caption{t distributed stochastic Neigherset Neighbour Spatial Embedding}
	\label{figure\arabic{figurecounter}}
\end{figure}
\begin{figure}
	\includegraphics[width=\maxwidth{\textwidth}]{figures/results_conductance_waves.png}
	\caption{t distributed stochastic Neigherset Neighbour Spatial Embedding}
	\label{figure\arabic{figurecounter}}
\end{figure}



\include{template/Optimization_of_existing_models_to_align_them_with_data.tex}
\include{template/Optimization_of_models_to_find_diverse_solutions_for_the_same_behaviors.tex}        
\include{template/Technical_details_of_optimizer.tex}
\include{template/Verification_of_the_optimizer.tex}
%<Insert your chapters here; I recommend to use
%\include{chapter2}                     %   \include rather than \input for chapters
%\include{chapter3}% etc.
                                        % Heading commands (in descending order):
                                        % \chapter
                                        % \section
                                        % \subsection
                                        % \subsubsection
                                        % \paragraph
                                        % \subparagraph
\iftoggle{sample}{%
  \chapter{Introduction}

\lipsum[1]

In this research effort, 
\section{Specific Aims}


\subsubsection{1A:} Optimize reduced neuronal models across four different cell types, using two different classes of models:
 
 
 \subsubsection{1B:} By fitting  cellular models to experimental data, using spike waveform shape features,  we can then take optimized models, and characterize and cellular behaviors using Allen SDK and EFEL feature extraction software to evaluate electrical robustness across a large number of features.	

\begin{centre} 
 \begin{tabular}{|c|c|c|c|}
 	\hline 
 	olfactory bulb mitral cell & layer IV pyramidal neuron & cerebellar purkinje cell & ca1 pyramidal cell \\ 
 	\hline 
 	& HH model score &  HH Model score&  HH model score & HH model score\\ 
 	\hline 
 	& Izhikitich model score  & Izhikitich model score & Izhikitich model score& Izhikitich model score \\ 
 	\hline 
 \end{tabular} 
\end{centre} 
 
\citep{searchinger_world_2013}
\refstepcounter{tablecounter}
 


\subsubsection{Aim 2:} Characterize and evaluate region-specific differences of cortical neuron-type data and models.
\subsubsection{2A:} Using large, publicly available datasets for neuron classes, we will characterize region-specific differences and evaluate neuron-type models for agreement with experimental data.
\subsubsection{2B:} Using models of neuron-types from different cortical regions we will identify the features underlying differences among neuron types and across regions, as well as the biophysical mechanisms that underlie those features. 

\subsection{This is the first sub-section-level heading}

\lipsum[1]

\subsubsection{This is the first sub-sub-section-level heading}

\lipsum[1]

\paragraph{This is the first paragraph-level heading}

\lipsum[1]

\subparagraph{This is the first sub-paragraph-level heading}

\lipsum[1]

\section{Citation examples}

The contents of this section differ depending on the bibliography settings, specifically whether the `usebiblatex' toggle is set to `true' or `false'.
\iftoggle{usebiblatex}{%
  This sentence shows citation with biblatex \parencite{searchinger_world_2013}.
  This is another sentence showing citation with biblatex \parencite{pathak_rural_2007}.
}{%
  This sentence shows citation with natbib \citep{pathak_rural_2007}.
  This is another sentence showing citation with natbib \citep{searchinger_world_2013}.
}

\section{Footnote examples}











\refstepcounter{tablecounter}%
}{}
%%%%%%%%%%%%%%%%%%%%%%%%%%%%%%%%%%%%%%%
% Back matter
%%%%%%%%%%%%%%%%%%%%%%%%%%%%%%%%%%%%%%%
\SingleSpacing                          % Back matter should be single spaced

\edef\defaulttolerance{\the\tolerance}
\tolerance 500                          % Increase tolerance to prevent material extending into margins
\hbadness 500

\iftoggle{useendnotes}{%                % If you're using endnotes, output them here
  \setsecnumdepth{none}                 % No section numbering in end notes
  \printpagenotes
  \setsecnumdepth{all}%                 % Turn section numbering back on after printing
}{}

\chapter*{\bibheading}                  % In the running text, use a chapter-level heading
                                        % for the bibliography section
\phantomsection
\addcontentsline{toc}{chapter}{%        % In the TOC, add a custom chapter-level heading
  \hspace{-\cftchapterindent}%          % that will be flush against the left margin
  \bibheading%
}
%\phantomsection
%\addtocontents{toc}%                    % Add this 'mark' to TOC so subsequent pages use
%  {\protect\markboth{\bibheading}{Page}}%   the bibliography heading (unlikely since
%                                        %   the appendices follow quickly)
\iftoggle{usebiblatex}{%                % Output the bibliography
  \printbibliography[heading=none]      % Using a 'biblatex' package; do not let
                                        %   'biblatex' output a heading
}{%
  \renewcommand\bibsection{}            % Do not let 'natbib' output a heading
  \bibliographystyle{\natbibstyle}      % Using 'natbib' to print bibliography
  \bibliography{\bibfilename}
}

\appendix                               % Indicate start of appendices
                                        % Appendices are considered 'mainmatter' in this
                                        %   documentclass
\tolerance \defaulttolerance            % Set tolerance back to default
\hbadness \defaulttolerance

\addtocontents{toc}{\protect%           % Only include appendix title in table of contents
  \setcounter{tocdepth}{0}}%            %   and omit sub-headings
\renewcommand*{\chapnamefont}%          % Reset font for 'Appendix' in chapter titles
    {\normalfont\MakeTextUppercase}
\makeatletter                           % Clear page after printing appendix title
  \renewcommand{\memendofchapterhook}%
  {%
    \clearpage
    \m@mindentafterchapter
    \@afterheading
  }
\makeatother

\phantomsection                         % Need '\phantomsection' to place hyperref
                                        %   bookmark more accurately
\addcontentsline{toc}{part}{Appendix}   %~Add "Appendix" to TOC here; comment out this
                                        %   line if you're not including appendices

%\phantomsection                        %!This is the one part of the template that I
%\addtocontents{toc}%                   %   could not get to work properly. After you
%  {\protect\markboth{APPENDIX}{Page}}  %   start listing appendices in the TOC,
                                        %   subsequent TOC pages should use "APPENDIX in
                                        %   the header instead of "CHAPTER"; however,
                                        %   this code will make "APPENDIX" appear on the
                                        %   the same page that the *first* appendix
                                        %   appears on. This problem won't affect most
                                        %   people, but if it affects you, uncomment
                                        %   these lines and move them below where
                                        %   the appendices are listed. Keep moving these
                                        %   lines down and checking the output until
                                        %   the TOC headers appear correctly

%\include{appendix1}                     %~Insert your appendices here; I recommend to use
%\include{appendix2}                     %   \include rather than \input for appendices.
%\include{appendix3}% etc.               %   All heading commands are the same as above,
                                         %   e.g., \chapter, \section, etc.
\iftoggle{sample}{%
  \include{template/sample_appendix}%
}{}

\backmatter                             % Start back matter according to documentclass
\makeatletter                           % Do not clear page after printing title for
  \renewcommand{\memendofchapterhook}%  %   biographical sketch
  {%
    \m@mindentafterchapter
    \@afterheading
  }
\makeatother
%\chapter{Biographical Sketch}           %~Biographical Sketch is optional
%\input{biography}                       %<Enter the name of the .tex file containing your
                                        %   biography or omit this line and type in
                                        %   your biography here (1 paragraph)
                                        
                                        


\end{document}
